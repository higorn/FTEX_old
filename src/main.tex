\documentclass[
	% -- opções da classe memoir --
	12pt,				% tamanho da fonte
	openright,			% capítulos começam em pág ímpar (insere página vazia caso preciso)
%	twoside,			% para impressão em recto e verso. Oposto a oneside
	oneside,
	a4paper,			% tamanho do papel.
	% -- opções da classe abntex2 --
	sumario=tradicional,
	chapter=TITLE,		% títulos de capítulos convertidos em letras maiúsculas
	section=TITLE,		% títulos de seções convertidos em letras maiúsculas
	%subsection=TITLE,	% títulos de subseções convertidos em letras maiúsculas
	%subsubsection=TITLE,% títulos de subsubseções convertidos em letras maiúsculas
	% -- opções do pacote babel --
	english,			% idioma adicional para hifenização
	french,				% idioma adicional para hifenização
	spanish,			% idioma adicional para hifenização
	brazil				% o último idioma é o principal do documento
	]{ftex}
% ---
% Pacotes básicos
% ---
\usepackage{lmodern}			% Usa a fonte Latin Modern
\usepackage[T1]{fontenc}		% Selecao de codigos de fonte.
\usepackage[utf8]{inputenc}		% Codificacao do documento (conversão automática dos acentos)
\usepackage{lastpage}			% Usado pela Ficha catalográfica
\usepackage{indentfirst}		% Indenta o primeiro parágrafo de cada seção.
\usepackage{color}				% Controle das cores
\usepackage{graphicx}			% Inclusão de gráficos
\usepackage{microtype} 			% para melhorias de justificação
\usepackage[table]{xcolor}

\usepackage{float}
\usepackage{multirow}
\usepackage[skip=0pt]{caption}
\usepackage{nomencl}
%\usepackage{subfigure}
\usepackage{pdflscape}
\usepackage{pdfpages}
\usepackage{xcolor}
% Definindo novas cores
\definecolor{verde}{rgb}{0.25,0.5,0.35}
\definecolor{jpurple}{rgb}{0.5,0,0.35}
% Configurando layout para mostrar codigos
\usepackage{listings}
\usepackage{chngcntr}
\newcommand{\estiloJava}{
    \lstset{
        language=Java,
        basicstyle=\ttfamily\small,
        keywordstyle=\color{jpurple}\bfseries,
        stringstyle=\color{red},
        commentstyle=\color{verde},
        morecomment=[s][\color{blue}]{/**}{*/},
        showspaces=false,
        showstringspaces=false,
        numbers=left,
        numberstyle=\tiny,
        breaklines=true,
        backgroundcolor=\color{cyan!10},
        breakautoindent=true,
        captionpos=t,
        xleftmargin=0pt,
        columns=fullflexible,
        frame=single,
        frameround=tttt,
        tabsize=2,
        belowskip=-12pt
    }
}
\newcommand{\estiloHtml}{
    \lstset{
        language=HTML,
        basicstyle=\small\ttfamily,
        keywordstyle=\color{black}\bfseries,
        breaklines=true,
        columns=fullflexible,
        frame=single,
        frameround=tttt,
        showstringspaces=false,
        tabsize=2,
        belowskip=-12pt
    }
}
\renewcommand{\lstlistingname}{Listagem}
\AtBeginDocument{\counterwithout{lstlisting}{section}}
% \usepackage[square,sort]{natbib}
% ---

% ---
% Pacotes adicionais, usados apenas no âmbito do Modelo Canônico do abnteX2
% ---
\usepackage{lipsum}				% para geração de dummy text
% ---

% ---
% Pacotes de citações
% ---
% \usepackage[brazilian,hyperpageref]{backref}	 % Paginas com as citações na bibl
\usepackage[alf,
    versalete,
    abnt-emphasize=bf,
    abnt-etal-list=3,
    abnt-etal-text=it,
    abnt-and-type=&,
    abnt-last-names=abnt,
    abnt-repeated-author-omit=no
]{abntex2cite}	% Citações padrão ABNT

% ---
% CONFIGURAÇÕES DE PACOTES
% ---

% ---
% Configurações do pacote backref
% Usado sem a opção hyperpageref de backref
% \renewcommand{\backrefpagesname}{Citado na(s) página(s):~}
% Texto padrão antes do número das páginas
\renewcommand{\backref}{}
% Define os textos da citação
% \renewcommand*{\backrefalt}[4]{
%	\ifcase #1 %
%		Nenhuma citação no texto.%
%	\or
%		Citado na página #2.%
%	\else
%		Citado #1 vezes nas páginas #2.%
%	\fi}%
% Espaçamento extra nas linhas de tabelas
\renewcommand{\arraystretch}{1.5}
% Fonte Arial
\usepackage{helvet}
\renewcommand{\familydefault}{\sfdefault}
% ---

% ---
% Informações de dados para CAPA e FOLHA DE ROSTO
% ---
\titulo{TÍTULO DO TRABALHO}
\autor{NOME DO AUTOR}
\local{Caxias do Sul}
\data{2018}
\orientador{Nome do orientador}
\coorientador{Nome do coorentador}
\instituicao{
\begin{figure}[!ht]
    \centering
    \includegraphics{img/logo-uniftec-azul-35x877.png}
\end{figure}
CURSO SUPERIOR DE ...}
\tipotrabalho{Trabalho acadêmico}
% O preambulo deve conter o tipo do trabalho, o objetivo,
% o nome da instituição e a área de concentração
\preambulo{Trabalho apresentado para o Curso de ...,
do Centro Universitário Uniftec, como parte dos requisitos para avaliação da unidade curricular de ...}
% ---

% ---
% Configurações de aparência do PDF final

% alterando o aspecto da cor azul
\definecolor{blue}{RGB}{41,5,195}

% informações do PDF
\makeatletter
\hypersetup{
     	%pagebackref=true,
		pdftitle={\@title},
		pdfauthor={\@author},
    	pdfsubject={\imprimirpreambulo},
	    pdfcreator={LaTeX with abnTeX2},
		pdfkeywords={abnt}{latex}{abntex}{abntex2}{trabalho acadêmico},
		colorlinks=true,       		% false: boxed links; true: colored links
    	linkcolor=blue,          	% color of internal links
    	citecolor=blue,        		% color of links to bibliography
    	filecolor=magenta,      		% color of file links
		urlcolor=blue,
		bookmarksdepth=4
}
\makeatother
% ---


% ---
% compila o indice
% ---
\makeindex
% ---

% ---
% compila a lista de siglas e abreveaturas
\makenomenclature
% ---

\begin{document}
% Seleciona o idioma do documento (conforme pacotes do babel)
%\selectlanguage{english}
\selectlanguage{brazil}

% Retira espaço extra obsoleto entre as frases.
\frenchspacing

% ----------------------------------------------------------
% ELEMENTOS PRÉ-TEXTUAIS
% ----------------------------------------------------------
\pretextual

% ---
% Capa
% ---
\imprimircapa
% ---
% ---

% Folha de rosto
% (o * indica que haverá a ficha bibliográfica)
% ---
\imprimirfolhaderosto*
% ---

% resumo em português
\setlength{\absparsep}{16pt} % ajusta o espaçamento dos parágrafos do resumo
\begin{resumo}

\lipsum[1] % remova o '\lipsum[1] e insira seu texto

\textbf{Keywords}: lorem. ipsum. dolor.

\end{resumo}

% resumo em inglês
\begin{resumo}[Abstract]
    \begin{otherlanguage*}{english}

\lipsum[2] % remova o '\lipsum[2] e insira seu texto

\textbf{Keywords}: lorem. ipsum. dolor.

    \end{otherlanguage*}
\end{resumo}

% ---
% inserir lista de ilustrações
% ---
\pdfbookmark[0]{\listfigurename}{lof}
\listoffigures*
\cleardoublepage
% ---

% ---
% inserir lista de tabelas
% ---
\pdfbookmark[0]{\listtablename}{lot}
\listoftables*
\cleardoublepage
% ---
 
% ---
% inserir lista de abreviaturas e siglas
% ---
\renewcommand{\nomname}{\listadesiglasname}
\pdfbookmark[0]{\nomname}{las}
% \begin{siglas}
%   \item[API] Application Programming Interface
%   \item[CRUD] Create, Read, Update e Delete
%   \item[MVC] Model View Control
% \end{siglas}
\makenomenclature
\printnomenclature
\cleardoublepage
% ---


% ---
% inserir o sumario
% ---
\pdfbookmark[0]{\contentsname}{toc}
%\dominitoc
\tableofcontents*
\cleardoublepage
% ---

 \captionsetup[table]{
  labelsep = newline,
  justification=justified,
  singlelinecheck=false,%%%%%%% a single line is centered by default
  labelsep=colon,%%%%%%
  skip = \medskipamount}
  
  
% ----------------------------------------------------------
% ELEMENTOS TEXTUAIS
% ----------------------------------------------------------
\textual

% ---
% Introdução
% !TEX root = ../main.tex
\chapter{INTRODUÇÂO}

\lipsum[2-10]

\section{Problema}

\lipsum[3-5]

\section{Hipóteses}

\lipsum[5]
\begin{itemize}
    \item Hipótese 1
    \item Hipótese 2
    \item Hipotese n
\end{itemize}

\section{Objetivo geral}

\lipsum[6]

\section{Objetivos Específicos}

\lipsum[7]
\begin{itemize}
    \item Objetivo 1
    \item Objetivo 2
    \item Objetivo n
\end{itemize}

\section{Justificativa}

\lipsum[7-20]

\section{Cronograma de Atividades}


% !TEX root = ../main.tex
\chapter{Desenvolvimento}

\section{Fundamentação teórica}

Para o \citeonline{autorx13}, esse é um exemplo de citação curta.

\lipsum[1-5]

\begin{citacao}
Exemplo citação longa. Fusce mauris. Vestibulum luctus nibh at lectus. Sed bibendum, nulla a faucibussemper, leo velit ultricies tellus, ac venenatis arcu wisi vel nisl. Vestibulum diam. Aliquampellentesque, augue quis sagittis posuere, turpis lacus congue quam, in hendrerit risuseros eget felis. Maecenas eget erat in sapien mattis porttitor. Vestibulum porttitor.
\end{citacao}

\lipsum[5-10]

\section{Metodologia}

\section{Resultados}

\include{textual/conclusao}

\postextual

% ---
% Bibliografia
\bibliography{bibliografia}
% ---

% ---
% Anexo 1 manual do usuário
% ---

% ---
% Anexo 2 Estratégia de testes
% ---

\end{document}

