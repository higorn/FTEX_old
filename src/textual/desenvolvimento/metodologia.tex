\section{Procedimentos metodológicos}
Essa sessão destina-se a descrever a metodologia usada no desenvolvimento da solução proposta, bem como a definir as métricas que serão usadas para obtenção dos dados e como estes serão analisados.

\subsection{Tipo de Pesquisa}
O presente trabalho, utiliza uma pesquisa exploratória e quantitativa, que visa estudar os diversos métodos e teorias, para desenvolver um sistema embarcado, para o monitoramento do consumo de energia elétrica de uma residência.

\subsection{Objeto de Estudo}
Os objetos de estudo desse trabalho, são as grandezas elétricas da Tensão e da Corrente, bem como as suas respectivas derivações, afim de compor uma série de dados, que após processados, representam as métricas de consumo.

\subsection{Coleta de Dados}
Os dados serão coletados a partir dos testes, realizados em ambiente controlado, através de sensores de corrente e tensão para um dispositivo que faz a leitura e envia para um outro dispositivo, via conexão serial, que faz o processamento dos dados.

\subsection{Análise dos Dados}
Os dados coletados nos testes, serão projetados em gráficos, afim de validar as informações e corrigir os possíveis pontos de falha.